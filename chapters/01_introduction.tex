\chapter{Introduction}

% Orientation: wider context

% situation
Thousands of articles are published every single day about the latest events happening.
The usage of specific language, the selection of details and how the narrative is presented, are all different aspects that are unique of each news outlet and author.
And these peculiarities, at the same time, can influence what the reader perceives about the events.
% framing definition
This whole set of information, may it be subtle or explicit, extends the raw facts that happen and is usually called \emph{Framing}~\cite{gamson1989media,scheufele1999framing}.
Framing can be created with any subjective statements that are mixed with the description of the event narrated, but does not stop there.
% TODO reference: ``Various observers have noted how subtle framing subtly and unconsciously [framing] operates'' (Gamson and Modigliani, 1989, p. 7)
Even selecting which details or features to report makes a big difference in the message sent to the reader.
% example
Taking for example two sentences ``\textit{the black man was shot}'' vs ``\textit{the man was shot}'', they have a different framing because, although the man shot was black, it is a judgement of the reporter whether this detail needs to be emphasised or not, implying a form of racism with respect to a race-independent murder.
Framing can have big impacts on the way readers perceive the content and relevance of the news~\cite{cohen2015press}. %(\textit{Agenda-setting}).


% Rationale: create a niche
% TODO why spotting is useful: http://faculty.sites.uci.edu/polletta/files/2016/02/22A-Simple-Intervention-to-Reduce-Framing-Effects-in-Perceptions-of-Global-Climate-Change22.pdf

% need of comparing multiple articles
Spotting the occurrence of framing is therefore a very difficult task, even for humans~\cite{morstatter2018identifying}. Something that could help in this situation is looking at different sources and analyse how they present the same event with different framing.
By seeing ``the other sides'' of the story we, as readers, could create a more complete picture and spot the differences at the macro (the perspective of the overall article) and micro-level (specific linguistic cues)~\cite{gamson1989media}.
The main problem of this technique is that it requires a lot of time,
and people can be very lazy while consuming the news~\cite{pennycook2019lazy}.
% to read, compare, track, and differentiate all the small details \todo{any citation to support this statement?}.
% tech limitations
At the moment, we can see a gap in the tools available to provide this functionality automatically.
Some technologies analyse parts of the problem, e.g. by grouping articles together by events (news aggregators), or anti-plagiarism tools that spot sentences occurring in multiple documents, or theoretical studies and conceptualisations analysing framing under different aspects.
But in our knowledge, none of them is is bringing together different stories to highlight the framing differences in an automatic way.


% Aim: purpose of research

This PhD aims at creating a methodology to extract and characterise framing differences among news articles.
This includes on one side revealing the choices done by the authors, and bring to the light the types of techniques they use to stand their point of view (e.g., selection and emphasis of details, addition of subjective content).
And on the other side, to study the information flow between sources and see which relationships exist between them. %(e.g., reusing content).
Given this aim, we target two Research Questions:

% Research Questions: short
\begin{itemize}
    \item RQ1: How can we automatically reveal the framing differences in articles presenting the same event?
    % \item RQ2: How do stories change between news sources? \todo{more about what}
    \item RQ2: How do news sources with different characteristics change stories over time?
    %RQ2: How can we identify patterns of information flow between news sources?
\end{itemize}

% RQ2: how stories change between news sources? (simpler)

% From the poster:
%How to automatically reveal the differences in stories about an event
%How to identify patterns of information flow between news sources?

% Method: methodology and theoretical framework

% Findings: outline of the findings

% Interpretation: general significance of the findings



% Document outline

The document is structured as follows.
In Chapter~\ref{chap:literature_review} we are providing a review of the literature available that analyses the sub-problems (framing analysis, similarity of documents).
Then, after revisiting and expanding the Research Questions in Chapter~\ref{chap:research_questions}, in Chapter~\ref{chap:proposal} we describe the methodology we plan to develop in order to tackle the research questions.% we analyse the articles with the several steps of the processing pipeline, describing the type of analysis we want to achieve.
% In Chapter~\ref{chap:evaluation} we present some directions for the evaluation and then
% we describe the research plan in Chapter~\ref{chap:plan} together with some early achievements.
In Chapter~\ref{chap:plan} we provide a timeline for the project, describing early achievements and a plan for the next two years of research.


% What, how and when
%Motivational (why)