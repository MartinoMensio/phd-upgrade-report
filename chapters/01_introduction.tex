\chapter{Introduction}

% Orientation: wider context

Situation: thousands of articles are published every day about events happening.
The usage of specific language, the selection of details and how the narrative is presented, are all different aspects that are unique of each news outlet/author.
And this peculiarities at the same time, can influence what the reader perceives about the events. They have big impacts.

% Rationale: create a niche

On the technological side, there is a gap of tools that can present the differences.
There are news aggregators, but they just group articles without saying what is different. There are tools to analyse plagiarism.
But no tool studies how the choices done by the writers relate between them.

% Aim: purpose of research

This PhD aims at characterising the choices done by authors, and to study the information flow between them.

% Research Questions: short
\begin{itemize}
    \item RQ1: How to automatically reveal the differences inn articles presenting the same event?
    \item RQ2: How to identify patterns of information flow between news sources?
\end{itemize}


% Method: methodology and theoretical framework

% Findings: outline of the findings

% Interpretation: general significance of the findings



% Document outline

The document is structured as follows.
In Chapter~\ref{chap:literature_review} I am providing a review of the fields related (TODO say something)
Then RQ expanded.
Then the proposal.
Some initial results.
The plan.


% What, how and when
%Motivational (why)