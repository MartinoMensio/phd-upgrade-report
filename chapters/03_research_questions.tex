\chapter{Research Questions}
\label{chap:research_questions}

% well-stated
% focused, concise
% feasible
% original
% at PhD level:  rigorous, publishable, sufficiently independent
% of appropriate scope

Given the gaps identified in the Literature Review, this PhD has the following Research Questions.

\vspace{12px}

\textit{\textbf{RQ1: How can we automatically reveal the framing differences in articles presenting the same event?}}

\vspace{12px}

This question focuses on how to develop a cross-article framing analysis that would overcome some of the limitations evidenced.
We can break it down in two subquestions:

\begin{itemize}
    % \item \textbf{RQ1.1: Which methods do we need to reveal the framing and narrative differences?} This subquestion deals with the methods that we need to retrieve, analyse, combine and explore the connections between news articles. % literature review
    
    \item \textbf{RQ1.1: Which cross-article signals can we define to express the framing differences?} We need to define some tangible signals that represent how the framing of the articles is emerging through linguistic devices. % experimentation
    % Which signals of framing can be identified from the comparison of multiple articles?
    
    % \item \textbf{RQ1.3: Which features reveal framing intentionality of linguistic choices?} Choosing terms sometimes is done with a specific intention to evoke or amplify a certain point, sometimes it is not.\todo{merge with RQ1.2?}
    %  To what extent do they reveal biases with respect to just capturing different ways of expressing the same events?
    
    \item \textbf{RQ1.2: How would a cross-article automatic framing analysis perform?} Answering this question is very important to understand how ready would the framework be to spot the signals identified previously. % usefulness
    
\end{itemize}


\vspace{12px}

% \textit{\textbf{RQ2: How do stories change between news sources?}}\todo{change}
\textit{\textbf{RQ2: How do news sources with different characteristics change stories over time?}}

\vspace{12px}

This second question instead moves the emphasis on the sources where articles are published. We can break it down in two subquestions:

\begin{itemize}
    \item \textbf{RQ2.1: How well does the framing analysis relate with the information available on affiliation, newsgroup and bias of the single outlets?} With different features available for news sources, we want to see if some of them are related to the amount and type of framing that occurs on their articles.
    
    \item \textbf{RQ2.2: To what extent can we identify cascades of information using temporal information?} We aim at extracting temporised chains that track details across news sources, from where they are seen the first time and how they evolve through time and sources.
    Can we identify some patterns that describe how the information flows between news sources?
    %How would this interact with rephrasing and other techniques that aim to avoid being spotted as plagiarism? (Linguistic variations vs semantics). And is temporal metadata reliable? (silent edits...)
    
    % \item \textbf{RQ2.2: What do these patterns tell us?} Do the patterns provide useful information and spot real % it has as assumption that there are patterns
    
    
\end{itemize}