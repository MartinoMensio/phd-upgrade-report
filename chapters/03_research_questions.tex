\chapter{Research Questions}
\label{chap:research_questions}

% well-stated
% focused, concise
% feasible
% original
% at PhD level:  rigorous, publishable, sufficiently independent
% of appropriate scope

Given the gaps identified in the Literature Review, this PhD has the following Research Questions.

\vspace{12px}

\textit{\textbf{RQ1: How can we automatically reveal the framing differences in articles presenting the same event?}}

\vspace{12px}

This question then breaks down into several subquestions:

\begin{itemize}
    \item \textbf{RQ1.1: Which methods do we need to reveal the framing and narrative differences?} This subquestion deals with the methods that we need to retrieve, analyse, combine and explore the connections between news articles.
    \item \textbf{RQ1.2: Which cross-article signals can we define to express the differences?} We need to define some tangible signals that represent how the framing of the articles is emerging through linguistic devices.
    \item \textbf{RQ1.3: Which features reveal framing intentionality of linguistic choices?} Choosing terms sometimes is done with a specific intention to evoke or amplify a certain point, sometimes it is not.\todo{merge with RQ1.2?}
    \item \textbf{RQ1.4: Would these differences be meaningful? To what extent do they reveal biases with respect to just capturing different ways of expressing the same events?} Answering this question is very important to evaluate the scope of application of this work. It is both an analysis of what is common and what is unique of each articles, but then we also add an analysis to extract which framing differences exist.   
\end{itemize}


\vspace{12px}

\textit{\textbf{RQ2: How do stories change between news sources?}}

\vspace{12px}

\begin{itemize}
    \item \textbf{RQ2.1: To what extent can we identify cascades of information by bringing together the analysis of RQ1 with temporal information?} We aim at extracting temporised chains, but how would this interact with rephrasing and other techniques that aim to avoid being spotted as plagiarism? (Linguistic variations vs semantics). And is temporal metadata reliable? (silent edits...)
    
    \item \textbf{RQ2.2: What do these patterns tell us?} Do the patterns provide useful information and spot real % it has as assumption that there are patterns
    
    \item \textbf{RQ2.3: How well does this analysis relate with the information available on News Groups, (country/location) and bias of the single outlets?}
    
\end{itemize}