\chapter{Timeline}
\label{chap:plan}
% Work Plan

This final chapter of this report describes the timeline, considering both the work to date and the plan for the next two years of research.
While the previous chapter described the methodology and evaluation, here we describe when each part of the project is expected to be addressed.

\section{Work to date}
% until now: what has been done
% - initial experiments
% - formalisation and problem definition
% - data

This first year can be divided into three tasks.


First is a set of \emph{practical experiments} that started by analysing the paper by~\citet{bountouridis2018explaining} which presents a methodology to analyse how much information overlaps between different similar documents, identifying points of information that are corroborated or omitted.
The implementation provided with the paper has been analysed and reproduced, to get a deep understanding of how it works.
Then, after finding and exploring methods that are better at finding the similarity between texts~\cite{cer2018universal,devlin2018bert,yang2019xlnet} by using language models, I experimented on how to use these methods to perform clustering both at the article level and at the sentence level.


sorting similarity

hierarchical clustering: example with diagram

sentence clusters: example with figure


Then the second task is the \emph{formalisation and definition} of the approach presented in Chapter~\ref{chap:proposal}.
From how to connect the different pieces together
And a position paper has been submitted, accepted and presented at a workshop on narrative analysis (Text2Story at ECIR) in April.
This position paper focused on describing the gap and the proposed cross-article signals that would show differences in how stories are narrated.


And finally the third task, \emph{data collection}, that has been done since the first steps on this projects.
The data that I am interested in belongs to different natures.
A wide set of articles is needed, so some datasets of news articles have been retrieved.
But also specific data that contains articles grouped by stories (document clustered) is very important to have, in order to have:
- a gold standard for document clustering experiments
- a solid starting point to analyse differences between articles that talk about the same events.

We have collected the following:
- allnews
- allsides
- google news

\section{Plan}

realistic
clearly stated milestones
dependencies explicit
contingency planning – risks identified
timeline – dates
resources
skills
pretty presentation

GANTT chart

